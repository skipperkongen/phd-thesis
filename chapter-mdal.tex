% !TEX root = ./thesis.tex
\chapter{Declarative Multi-scale Visualization of Large Graphs}

% Define clear usecase
% - Zoomable visualizations of large social networks

We want to:
\begin{itemize}
\item Model all data as graphs, because graphs are general (they subsume spatial, temporal and social data). 
\item Analyze graphs using zoomable graph visualizations
\item Share visualizations with the world
\item Support declarative languages for creating zoomable graph visualizations
\end{itemize}

We assume that there:

\begin{itemize}
\item Exists methods for translating any data into nodes and edges form; we will give examples
\item Exists methods for embedding nodes and edges in the plane, i.e. computing two-dimensional coordinates for nodes and edges
\end{itemize}

For this to work, we must:
\begin{itemize}
\item Define a notion of \emph{resolution} or \emph{scale} for graphs; it is sufficient to define a unit for the two-dimensional coordinate system, e.g. meters
\item Define a query model for graph views, e.g. the pyramid model because it is scalable
\item Define metrics for evaluating graph views, e.g. information density metrics (see Woodruff et al.~\cite{woodruff1998constant})
\item Define valid transformation functions for turning one graph into another graph

\item Formulate constraints that can be evaluated over graph views
\item Define a distance function 
\item Define an equivalence relation between graph views at different scales, i.e. what are the constraints that must hold when transforming graphs between scales

\item Define metrics for evaluating scale-based views for graphs, e.g. density metric


We need to guide the abstraction process, i.e. objective functions. 

We want the multi-scale data abstraction of graphs to be user-programmable, to support data enthusiasts and professional analysis. We need scalable algorithms and systems that implement the whole deal.

\end{itemize}

% Motivation

\section{Introduction}

\section{Multi-scale Graph Abstraction}
We want to compute a multi-scale data abstraction for a graph. at a set of discrete scales. For this we need to: Model data (i.e. spatial, temporal, social data) as graphs. Define a scale-based view of graphs (projection of graph into metric space, representation of metric space in pixel space). Define equivalence classes for graphs (subject to constraints)
\subsection{Modeling Data Domains}
% how can we model different domains as graphs
% - spatial
% - temporal
% - 
\subsection{Scale-based View Model}

\subsection{Data Abstraction Operators}
\subsection{Equivalent Graphs}
\paragraph{Graph Minor}
\paragraph{Constraints}


\section{Modeling}
\subsection{Data Model}
\subsection{Query Model}
\subsection{Graph Projection}
\subsection{Scale-based Equivalence}

\section{Related Work}

\section{Conclusion}