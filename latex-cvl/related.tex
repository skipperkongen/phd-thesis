% !TEX root = ./ICDE14_conf_full_296.tex
\section{Related work}
\label{sec:related}

Cartographic generalization is a classic topic of investigation in the GIS community, and several models have been developed for generalization operations~\cite{harrie2007modelling}. \hl{While} the problem has been considered by some as AI complete~\cite{frank1994multiscaletree}, recent work has focused on automatic map generalization based on optimization models or queries for filtering~\cite{sarma2012fusiontables,nutanong2012multiresolution}. This reduction in scope reflects the need of providing a wide variety of web-accessible maps summarizing ever increasing amounts of geospatial datasets. Our work provides support for the same trend.  

The optimization approach of Das Sarma et al.~\cite{sarma2012fusiontables} is the most related to our work. In contrast to our approach, however, Das Sarma et al. do not provide a flexible declarative interface for user-defined constraints, nor does their approach leverage SQL. 
%As a consequence, their approach is restricted to memory-resident datasets and pre-specified, not user-defined, constraints. 
In addition, it is hard to integrate their approach with existing geospatial data serving infrastructures, which are mostly based on standard spatial database \hl{technology}.

User-defined constraints and declarative specifications have been shown to yield elegant solutions to a variety of problems in data management, including record deduplication~\cite{ArasuRS09:Dedupalog}, database testing~\cite{BinnigKL07:ReverseQP,Binnig:2007:SymbolicQP}, as well as cloud and networking resource optimization~\cite{Liu:2012:Cologne}. Our work brings these ideas to the context of map generalization and geospatial data, and as mentioned previously, is among the first frameworks to implement the vision of reverse data management~\cite{meliou2011reverse}. 

In-database processing has also been explored successfully in diverse contexts in the literature. Translation of high-level languages, such as XQuery or LINQ, to SQL lead to highly scalable and efficient implementations~\cite{pathfinder,ferry}. A number of recent approaches have targeted expressing complex statistical analytics in SQL~\cite{Hellerstein:2012:Madlib,Ordonez2007:StatisticsUDFs}. In contrast, we show how in-database processing can be used in the implementation of a declarative language for map generalization which includes solvers and constraints, leveraging the trend to incorporate whole programming language interpreters and support for spatial data structures in database engines~\cite{Blakeley2008:DotNET}.  

Our approach dovetails with a number of techniques from the literature, which hold potential to further extend or complement it. \hl{First, we observe that the running time of the LP-based greedy algorithm (LPGA) is generally high. We implemented this algorithm because it provides a theoretical bound on the solution quality. We plan to explore other algorithms for set multicover, such as the greedy algorithm described by Rajagopalan and Vazirani}~\cite{rajagopalan1998primal}\hl{, to improve running time compared to the LP-based greedy algorithm, while achieving good quality. An interesting challenge is how to express such algorithms entirely in SQL.}

\hl{Second, this work considers only selection of objects. An important extension is to allow other data reduction operations, such as geometric transformation and aggregation of objects. While we believe that CVL could be adapted to these additional requirements, this would imply modeling alternative semantics and procedures for satisfying constraints in our framework.}

\hl{Third, we would like to experiment with geospatially-aware parallel processing infrastructures, such as Hadoop-GIS}~\cite{Aji:2013:HadoopGIS}\hl{, for even further scalability in computing map generalizations. Finally, once a map generalization is complete, the resulting map must be served to end-users. This problem is orthogonal to our work, and classic linearization techniques can be applied}~\cite{hilbert1891ueber}\hl{. All of these are interesting avenues for future work.}
