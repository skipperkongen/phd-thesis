\documentclass[11pt, oneside]{report}   	% use "amsart" instead of "article" for AMSLaTeX format
\usepackage{geometry}                		% See geometry.pdf to learn the layout options. There are lots.
\geometry{letterpaper}                   		% ... or a4paper or a5paper or ... 
%\geometry{landscape}                		% Activate for for rotated page geometry
%\usepackage[parfill]{parskip}    		% Activate to begin paragraphs with an empty line rather than an indent
\usepackage{graphicx}				% Use pdf, png, jpg, or eps with pdflatex; use eps in DVI mode
								% TeX will automatically convert eps --> pdf in pdflatex		
\usepackage{amssymb}

% Copied from Marcos Style:
\setlength{\parindent}{0px}
\setlength{\parskip}{1em}

\title{Declarative Design and Efficient Serving of Web-Based Zoomable Maps}
\author{Pimin Konstantin Kefaloukos}
%\date{}							% Activate to display a given date or no date

\begin{document}
\maketitle

\tableofcontents

\chapter{Introduction}

\section{Motivation}
% Why are MV important? Because they enable insights and decision making on the go.
Web-based multiscale visualizations are important, because they enable a visual means to \emph{insights} and \emph{decision making} on the go. Specifically, multiscale visualizations enable information to be explored at different focal points and different levels of abstraction. For example, a train passenger can read news articles enriched with data-driven visualizations, which facilitates insights into complex phenomena (e.g. the economy, society or culture). A tourist walking about in a city can consult a mobile map to search for amenities (e.g. a nice restaurant), which facilitates decision making.

As a consequence of near-ubiquitous web access, and the rising popularity of mobile devices, users can spontaneously engage web-based visualizations anywhere and anytime. In combination with the exponential growth in data online~\cite{foo,foo,foo}, web-based visualizations (e.g. zoomable web maps) have recently attracted much larger audiences and can potentially integrate much larger and constantly evolving repositories of data~\cite{gst, foo}.


Furthermore, 
This poses three high-level challenges to map designers and system engineers. First and second, map designers must address the dual problems of \emph{data abstraction} (e.g. selection and aggregation of information) and \emph{visual abstraction} (e.g. choice of information styles)~\cite{stolte2003multiscale}. While these problems are referred to as \emph{model generalization} and \emph{cartographic generalization} in the cartographic literature~\cite{foerster2007towards}, we use the terms used in the database visualization literature. Third, system engineers must cost-effectively \emph{manage query workloads} as online users engage a map application.

\section{Data Abstraction}

\section{Serving Geographical Maps}

\bibliographystyle{plain}
\bibliography{thesis}
                             % Sample .bib file with references that match those in
                             % the 'Specifications Document (V1.5)' as well containing
                             % 'legacy' bibs and bibs with 'alternate codings'.
                             % Gerry Murray - March 2012



Marcos ambitious idea, three parts:
- (I) Introduction: talk about the two chapters that follow. 
- (P1) Pt 1: bulk: declarative cartography, about PRODUCING maps
- (P2) Pt 2: about SERVING digital maps
- Similar structure for both: state of the art, my contributions (advancing the state of the art).  

What-goes-where:
- I <- Case Study, P1 enables P2, Mention where related work is (which parts/chapters)
- P1 <- 
        Survey generalization, 
        CVL1 (closing the gap, part a), 
        CVL2 (closing the gap, part b)
- P2 <- 
        Survey on serving maps, getting to keyed data for maps, all this stuff you can use 
                incl. vectile (step 2),
                caching,
                prediction,
                data partitioning (step 3), 
                replication (step 4), 
                consistency, 
                key-value stores, 
                classic web serving infrastructures,
                papers from seminar
        Gap between state of the art, and the agency
        TileHeat (closing the gap, step 1), deploying a caching infrastructure
        
Publication strategy:

CVL2 (it's a journal thing, not a conference submission):
Approach: "Make CVL more general (star approach), with comparable performance"
- GeoInformatica (extended version of CVL), "completeness/thoroughness": repeat experiments, show new use cases (comp. to CVL), illustrate CVL2 with the new use cases 
- Only conference if "technical fun" or "something new" in compiler, real tech. take-away

\end{document}  

