\documentclass[11pt, oneside]{report}   	% use "amsart" instead of "article" for AMSLaTeX format
\usepackage{geometry}                		% See geometry.pdf to learn the layout options. There are lots.
\geometry{letterpaper}                   		% ... or a4paper or a5paper or ... 
%\geometry{landscape}                		% Activate for for rotated page geometry
%\usepackage[parfill]{parskip}    		% Activate to begin paragraphs with an empty line rather than an indent
\usepackage{graphicx}				% Use pdf, png, jpg, or eps with pdflatex; use eps in DVI mode
								% TeX will automatically convert eps --> pdf in pdflatex		
\usepackage{amssymb}

\title{A Declarative Language for Multiscale Data Abstraction of Spatial Datasets}
\author{Pimin Konstantin Kefaloukos}
%\date{}							% Activate to display a given date or no date

\begin{document}
\maketitle

\tableofcontents

\part{Introduction}

\chapter{Multiscale Data Visualizations}

\section{History}
On a cave wall in Lascaux, France, a series of dots dating to 16.500 BC have captured the imagination of scholars. These ancient dots are arranged so much like well-known constellations, that some scholars believe them to be a celestial map. On a clear night, the stone-age sky above Lascaux would have contained thousands of visible stars; clearly the artist did not paint them all. Before setting to work, the artist made two key decisions, namely \emph{which} stars to paint and \emph{how} to paint them. While the latter problem was addressed by painting dots on a cave wall, the first problem -- choosing which stars to paint -- was perhaps addressed by selecting the brightest stars and most prominent constellations on the night sky. In the field of data visualization, these problems are known respectively as \emph{data abstraction} (e.g. selection of stars) and \emph{visual abstraction} (e.g. painting stars as dots)~\cite{stolte2003multiscale}.

A few thousand years later, man turned his attention to the ground beneath his feet. In Mesopotamia, clay tablets dating to 2.500 BC depict what is essentially a cadastral map~\cite{harley1987cartography}. The purpose of a cadastral map is to correctly identify land owners, not to convey accurate information about physical reality. So, if the physical borders of the land parcels are cleared marked with stone-walls, a cadastral map only needs to be \emph{topologically} correct.

\section{Logical zoom}
Cave wall and clay tablets support only a primitive concept of zooming. When the spectator moves towards or away from surface, the objects seemingly become larger or smaller. By contrast, a computer screens enables a more sophisticated type of \emph{logical zoom}~\cite{van1990reactive}. In response to a logical zoom, a new data abstraction and visual abstraction can be computed and presented to the spectator. The abstraction can be adapted to the available data and the screen size. For example, data points can be selected or aggregated, and can be drawn differently. 

\section{Principles}
In 1961, Friedrich T\"{o}pfer formulated the \emph{Principle of Selection} for the field of cartography~\cite{topfer1966principles}. The principle has a slightly technical formation, however the core idea is that maps at different scale should have the same amount of information per unit display area (e.g. per $cm^2$ on a sheet of paper). In 1998, Allison Woodruff et al. reformulated it as the \emph{Principle of Constant Information Density}, in the context of zoomable visualizations on computer screens~\cite{woodruff1998constant}. The original principle, as formulated by T\"{o}pfer, has been verified empirically, e.g. by counting the number of settlements shown per unit display area on different hand-crafted maps of Scotland~\cite{topfer1966principles}. 

It is not difficult to understand why the Principle of Constant Information Density must be true. Screens have a fixed number of pixels for rendering objects, and similarly, sheets of paper have a fixed area on which to place ink. There is a physical bound on how much information can be represented per unit display area (e.g. a few thousand pixels per $cm^2$ on a computer screen). Simultaneously, there is a lower bound on how small things can get and still remain legible to the human eye.

In 2012, Das Sarma et al. published a paper that listed the requirements for multiscale map visualizations supported by the commercially available Google Fusion Tables system~\cite{sarma2012fusiontables}. One requirement is that visualizations must satisfy a \emph{zoom-consistency} constraint; when a user zoom in, new objects may appear on the map, but must never disappear. Mathematically, if $S_z$ is the set of objects selected at scale $z$, then $S_z \subseteq S_{z'}$ for all $z < z'$. Intuitively, you must never loose sight of an object when zooming in on it.

Another constraint mentioned by Das Sarma et al. is the \emph{visibility} constraint, which is closely related to the Principle of Constant Information Density. This constraint makes the general principle more specific by bounding the number of objects that are visible within a unit display area.

%Back when the earth was believed to be flat, assuming that Eratosthenes was not around to suggest otherwise, 

\section{Modern-age Data Visualization}

\chapter{Multiscale Geographical Maps}
% section
Illustrate importance of multiscale visualization via reference to information cartography, data cube visualization, evolutionary timelines, social network visualizations and digital maps.

% section
\section{Principles}
Mention Weibel definition (salience etc). Reference woodruff and toepfer (constant information density). Reference da sarma (zoom consistency and visibility constraint). Reference samet (proximity constraint). Reference someone for topological constraints.
% section

\section{Multiscale Abstraction}
Define difference between data- and visual abstraction, reference data cube paper. Reuse points and figures from ICDE 2014 presentation (what to display versus how to display). Mention the geographical terms of model generalization and cartographic generalization (reference Gruenreich). 

\subsection{Data abstraction}
Mention that data abstraction for maps is surveyed in Part I chapter 1.

\subsubsection{}

\subsection{Visual abstraction}
Reference visual variables.

\section{Case Study: Danish Geodata Agency}
Mention GST and their use cases (POI and place names).

\chapter{Map Services}

\section{Map Services}
Talk about how map services are used.

\section{Case Study: Danish Geodata Agency}
Mention GST and their use cases (POI and place names).


\part{Producing Maps}

\chapter{Data abstraction methods for maps}
Survey the literature.

Reference Weibel and Harrie (constraint-based approach and salience). Reference Wollf (land use IP). Reference Da Sarma (fusion tables IP).


\chapter{}


\part{Serving Maps}

\bibliographystyle{plain}
\bibliography{thesis}
                             % Sample .bib file with references that match those in
                             % the 'Specifications Document (V1.5)' as well containing
                             % 'legacy' bibs and bibs with 'alternate codings'.
                             % Gerry Murray - March 2012



Marcos ambitious idea, three parts:
- (I) Introduction: talk about the two chapters that follow. 
- (P1) Pt 1: bulk: declarative cartography, about PRODUCING maps
- (P2) Pt 2: about SERVING digital maps
- Similar structure for both: state of the art, my contributions (advancing the state of the art).  

What-goes-where:
- I <- Case Study, P1 enables P2, Mention where related work is (which parts/chapters)
- P1 <- 
        Survey generalization, 
        CVL1 (closing the gap, part a), 
        CVL2 (closing the gap, part b)
- P2 <- 
        Survey on serving maps, getting to keyed data for maps, all this stuff you can use 
                incl. vectile (step 2),
                caching,
                prediction,
                data partitioning (step 3), 
                replication (step 4), 
                consistency, 
                key-value stores, 
                classic web serving infrastructures,
                papers from seminar
        Gap between state of the art, and the agency
        TileHeat (closing the gap, step 1), deploying a caching infrastructure
        
Publication strategy:

CVL2 (it's a journal thing, not a conference submission):
Approach: "Make CVL more general (star approach), with comparable performance"
- GeoInformatica (extended version of CVL), "completeness/thoroughness": repeat experiments, show new use cases (comp. to CVL), illustrate CVL2 with the new use cases 
- Only conference if "technical fun" or "something new" in compiler, real tech. take-away

\end{document}  

