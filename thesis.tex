\documentclass[11pt, oneside]{report}   	% use "amsart" instead of "article" for AMSLaTeX format
\usepackage{geometry}                		% See geometry.pdf to learn the layout options. There are lots.
\geometry{letterpaper}                   		% ... or a4paper or a5paper or ... 
%\geometry{landscape}                		% Activate for for rotated page geometry
%\usepackage[parfill]{parskip}    		% Activate to begin paragraphs with an empty line rather than an indent
\usepackage{graphicx}				% Use pdf, png, jpg, or eps with pdflatex; use eps in DVI mode
								% TeX will automatically convert eps --> pdf in pdflatex		
\usepackage{amssymb}

% Copied from Marcos Style:
\setlength{\parindent}{0px}
\setlength{\parskip}{1em}

\title{Declarative Design and Efficient Serving of Zoomable Geographical Maps}
\author{Pimin Konstantin Kefaloukos}
%\date{}							% Activate to display a given date or no date

\begin{document}
\maketitle

\tableofcontents

\chapter{Introduction}

Case Study, P1 enables P2, Mention where related work is (which parts/chapters)


\section{Motivation}
% Why are they important? Use cases.
Geographical maps are valuable because they facilitate spatial insights and enable spatial decision making. For example, a journalist writing a story about a war may choose to enrich the article with a map visualization, which allows readers to gain spatial insights about the location of troops, resources and refugees. As another examples, a tourist wandering the streets of an unfamiliar city may choose to use a location based service (LBS) to locate a near-by restaurant, which is an example of spatial decision making. Importantly, maps should be made \emph{zoomable} whenever data can not be legibly visualized within a single frame. This is typically the case. Zoomable maps allows users to explore vast amounts of data at different focal points and levels of abstraction. For example, points can be selected and aggregated at low scales, and fully shown only at high scales, which enables switching between overview mode and detailed exploration.

% What are maps and how are they used?
Web-based multiscale maps have recently attracted millions (or even billions) of users who are attracted to web maps by a fruitful combination of several trends. In recent years, there has been an explosive growth in the amount and availability of useful spatial data (e.g. commercial, environmental, political and social data), driven in part by the availability of effective geocoding services. Furthermore, web access is now close to being ubiquitous in many parts of the world. Finally, as world-wide sales of mobile devices (e.g. smart phones, tablets and laptops) are soaring, millions of users now have the ability and motivation to spontaneously engage maps -- anywhere, anytime and for many purposes.

% challenges: caused by high traffic and constant surfacing of new and often big datasets.
This growth in users and data poses three key challenges to map designers and system engineers. First and second, map designers must frequently and rapidly address the problems of multiscale \emph{data abstraction} (e.g. selection and aggregation) and \emph{visual abstraction} (e.g. choice of graphical style) as new zoomable maps are needed and new datasets are integrated~\cite{stolte2003multiscale}. Third, system engineers must cost-effectively \emph{manage query workloads} as millions of online users engage a map application.




\section{Principles and Requirements for Geographical Maps}
\subsection{Basic Principles}
\subsection{Requirements as Stated by the Industry}
Apple iPhone 6 maps requirements listed in SELECT-DISTINCT paper, and in duking it out paper. Constant Information density

\section{Zoomable Geographical Maps on the Web}
\subsection{Definitions}
\subsection{Key Challenges}

\section{Overview of State-of-the-Art in Geographical Maps on the Web}
\subsection{Vector Tiles and Client-side Rendering}


\section{Research Gaps}
\subsection{Declarative Design}
\subsection{Workload Prediction}

\section{Contributions of the Dissertation}

\subsection{CVL1}
Brief summary and reference to Section.
\subsection{CVL2}
Brief summary and reference to Section.
\subsection{TileHeat}
Brief summary and reference to Section.


\part{Automatic Multiscale Data Abstraction}

\chapter{Survey of Automatic Multiscale Data Abstraction}

\section{Online processing}

\section{Preprocessing}

\chapter{CVL1}
Closing the gap, part a

\chapter{CVL2} 
Closing the gap, part b

\part{Serving Web Maps}

\chapter{Survey of Map Serving}

\section{Getting to Keyed Data for Maps}

\section{Caching}

\section{Prediction}

\section{Data Partitioning}

\section{Replication}

\section{Consistency}

\section{Key-value Stores}

\section{Classic Web Serving Infrastructures}

\chapter{Case Study: Danish Geodata Agency}

\section{Gap between the State of the Art, and the Agency}

\chapter{TileHeat}
TileHeat (closing the gap, step 1), deploying a caching infrastructure


\bibliographystyle{plain}
\bibliography{thesis}
                             % Sample .bib file with references that match those in
                             % the 'Specifications Document (V1.5)' as well containing
                             % 'legacy' bibs and bibs with 'alternate codings'.
                             % Gerry Murray - March 2012



Marcos ambitious idea, three parts:
- (I) Introduction: talk about the two chapters that follow. 
- (P1) Pt 1: bulk: declarative cartography, about PRODUCING maps
- (P2) Pt 2: about SERVING digital maps
- Similar structure for both: state of the art, my contributions (advancing the state of the art).  

What-goes-where:
- I <- Case Study, P1 enables P2, Mention where related work is (which parts/chapters)
- P1 <- 
        Survey generalization, 
        CVL1 (closing the gap, part a), 
        CVL2 (closing the gap, part b)
- P2 <- 
        Survey on serving maps, getting to keyed data for maps, all this stuff you can use 
                incl. vectile (step 2),
                caching,
                prediction,
                data partitioning (step 3), 
                replication (step 4), 
                consistency, 
                key-value stores, 
                classic web serving infrastructures,
                papers from seminar
        Gap between state of the art, and the agency
        TileHeat (closing the gap, step 1), deploying a caching infrastructure
        
Publication strategy:

CVL2 (it's a journal thing, not a conference submission):
Approach: "Make CVL more general (star approach), with comparable performance"
- GeoInformatica (extended version of CVL), "completeness/thoroughness": repeat experiments, show new use cases (comp. to CVL), illustrate CVL2 with the new use cases 
- Only conference if "technical fun" or "something new" in compiler, real tech. take-away

\end{document}  

