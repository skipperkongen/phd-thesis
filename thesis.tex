\documentclass[11pt, oneside]{report}   	% use "amsart" instead of "article" for AMSLaTeX format
\usepackage{geometry}                		% See geometry.pdf to learn the layout options. There are lots.
\geometry{letterpaper}                   		% ... or a4paper or a5paper or ... 
%\geometry{landscape}                		% Activate for for rotated page geometry
%\usepackage[parfill]{parskip}    		% Activate to begin paragraphs with an empty line rather than an indent
\usepackage{graphicx}				% Use pdf, png, jpg, or eps with pdflatex; use eps in DVI mode
								% TeX will automatically convert eps --> pdf in pdflatex		
\usepackage{amssymb}
\usepackage{amsmath}

% Copied from Marcos Style:
\setlength{\parindent}{0px}
\setlength{\parskip}{1em}

\title{Declarative Multi-Scale Selection and Scalable Serving of Geospatial Datasets}
\author{Pimin Konstantin Kefaloukos}
%\date{}							% Activate to display a given date or no date

\begin{document}
\maketitle

\tableofcontents

% STUART: Identifying the problem and objectives
\chapter{Introduction}

%Case Study, P1 enables P2, Mention where related work is (which parts/chapters)

% Emphasise map production and map consumption
\section{Motivation}
% Opportunities for users
\emph{Pannable} and \emph{zoomable} geographical maps are important because they allows users to explore large repositories of information at different focal points and levels of abstraction. Furthermore, \emph{web-based} maps enable online users world-wide to gain spatial insights and make spatial decisions on the go. For example, journalists can enrich online articles with digital maps based on the constantly emerging datasets online. Such maps give readers spatial insight into news stories with geographical aspects: the location of troops in a war, the extent of a pollution incident, places worth traveling to, etc. As another example, tourists can use digital maps and location based services (LBS) on mobile devices while exploring an unfamiliar city. Such maps support decision making: finding near-by restaurants, shops, points of interest, etc. Zoomable web maps have have many use cases, e.g. in social media, real-estate, government, science, policing, warfare, health, education and environmental control.%Finally, real-estate agents can advertise new deals on geographical maps, which is of great interest to people looking to buy or sell homes in a particular area. % work with the examples

% Challenge 1: too many users
One challenge is encountered during the serving of maps. Today, users of maps have the opportunity to engage maps anytime and anywhere, caused by advances in mobile technology and near-ubiquitous web access in many parts of the world. As a consequence, web-based maps have attracted millions of users who bombard the end-points of spatial data services with a crushing request workload around the clock -- especially during peak hours. To handle this workload there is a need for scalable system architectures for maps. %TODO: Explain the challenges presented to system engineers further... and give some examples.

% Challenge 2: big data
Another challenge is encounted during the production of maps. Everyday, a mass of spatial data is accumulating online~\cite{agrawal2012bigdata}. There is a clear need to visualize much of this data on maps for the reasons mentioned above. However, manually editing maps based on emerging big data is prohibitively time-consuming. Furthermore, relying on GIS specialists to handle the task is expensive at best. What is needed is efficient and effective methods that can be used by non-experts who must perform tasks such as \emph{data abstraction} (e.g. data selection and aggregation)~\cite{haunert2006landcover,schmid2013opensciencemap} and \emph{visual abstraction} (e.g. graphical expression)~\cite{jacques1967semiologie} in order to create useful multi-scale maps~\cite{stolte2003multiscale,weibel1999generalising}. 

%These processes are made much harder in the context of Big Data. Furthermore, these tasks generally involve either preprocessing~\cite{sarma2012fusiontables,kefaloukos2014declarative} or indexing~\cite{bereuter2013real,nutanong2012multiresolution} of the data, which also becomes harder for Big Data. TODO: Give some examples

\section{Overview of State-of-the-Art Online Multi-scale Maps}
\subsection{Multi-scale Data Abstraction}
\subsection{Vector Tiles}

\section{Requirements for Online Multi-scale Maps}
\subsection{Map Design Objectives}
% toepfer, constant info dens. metrics
\subsection{System Objectives}
% high-availability (basically no down-time), high-performance, scalable
\subsection{Usability Objectives}
% non-expert usage, time-efficient, effective
\subsection{Case Study: Danish Geodata Agency}
% authoritative data. consistent data. high-availability. high-performance. standards. cost-effective.

\section{Research Gaps}
\subsection{Programmability}
\subsection{Moving code to data}
\subsection{Unified Model for Multi-scale Data Visualization}
\subsection{Scalability}
\subsection{Resource Optimization}

% abstraction: usable by non-experts, programmability, unifying data abstraction across domains (addressed by CVL2)
% Vector Tiles: prediction

\section{Contributions}
\subsection{Design of Declarative Languages for Multi-scale Data Abstraction of Relational Datasets}
\subsection{Compilers for Multi-scale Data Abstraction Languages}
\subsection{Prediction and Tile Selection for Geospatial Workloads}
\subsection{Publications}

\section{Structure of this Dissertation}


\bibliographystyle{plain}
\bibliography{thesis}
                             % Sample .bib file with references that match those in
                             % the 'Specifications Document (V1.5)' as well containing
                             % 'legacy' bibs and bibs with 'alternate codings'.
                             % Gerry Murray - March 2012



Marcos ambitious idea, three parts:
- (I) Introduction: talk about the two chapters that follow. 
- (P1) Pt 1: bulk: declarative cartography, about PRODUCING maps
- (P2) Pt 2: about SERVING digital maps
- Similar structure for both: state of the art, my contributions (advancing the state of the art).  

What-goes-where:
- I <- Case Study, P1 enables P2, Mention where related work is (which parts/chapters)
- P1 <- 
        Survey generalization, 
        CVL1 (closing the gap, part a), 
        CVL2 (closing the gap, part b)
- P2 <- 
        Survey on serving maps, getting to keyed data for maps, all this stuff you can use 
                incl. vectile (step 2),
                caching,
                prediction,
                data partitioning (step 3), 
                replication (step 4), 
                consistency, 
                key-value stores, 
                classic web serving infrastructures,
                papers from seminar
        Gap between state of the art, and the agency
        TileHeat (closing the gap, step 1), deploying a caching infrastructure
        
Publication strategy:

CVL2 (it's a journal thing, not a conference submission):
Approach: "Make CVL more general (star approach), with comparable performance"
- GeoInformatica (extended version of CVL), "completeness/thoroughness": repeat experiments, show new use cases (comp. to CVL), illustrate CVL2 with the new use cases 
- Only conference if "technical fun" or "something new" in compiler, real tech. take-away

\end{document}  

